\documentclass[a4paper,20pt,twoside]{report}
\usepackage[left = 0.5in, top = 0.5in,bottom = 0.5in]{geometry}
\usepackage{titlesec}
\pagestyle{myheadings}

\begin{document}
	\title{Simulating Motion Of Charged Particle/Electron Through an Electric and Magnetic Field}
	\maketitle
	
	\section[15pt]{Aim}
	To simulate the motion of a charged particle inside an Electric and Magnetic Field.
	
	\section[15pt]{Introduction}
	The project focuses on simulating/tracing the motion of a charged particle/electron
	under the influence of an external Electric and Magnetic Fieled.
	The electron trajectories can be given an desired shape by applying an Electric or
	Magnetic Field
	
	\section[15ppt]{Applications}
		\begin{description}
			\item [$\bullet$ Radio]
			\item [$\bullet$ Television]
			\item [$\bullet$ CRO]
			\item [$\bullet$ Mass Spectrograph]
			\item [$\bullet$ Particle Accelerators]
	    \end{description}	
    
    \section{Motion of electron inside an uniform Electric Field}
    \subsection{Types of motion exhibited by electron}
    \begin{description}
    	\item [$\bullet$ Parallel to Electric Field]
    	\item [$\bullet$ Perpendicular to Electric Field]
    	\item [$\bullet$ Projected at an angle to the Electric Field]
    \end{description}
		
	\subsection{Basic Calculation}
		When an electron is projected inside an electric field, sustained by 2 plates
		\textbf{A} and \textbf{B}
		with A being 
		\textbf{positively}
		charged and B being
		\textbf{negatively}
	    charged, having a potential
		\textbf{V}
		 given by
		\textbf{[$V_a$ - $V_b$]}
		at a distance of \textbf{d}
		 the intensity of the electric field between the plate is given by
		
		\begin{equation}
		E = \frac{V}{d}
		\end{equation}
		
		
		Force experienced by an electron at rest on plate
		\textbf{B}
		\begin{equation}
		F = -eE
		\end{equation}
		
		\textnormal{The acceleration is given by}
		\begin{equation}
		a = \frac{F}{m} = -\frac{eE}{-eE}{m}
		\end{equation}
		\vspace{10mm}
		\textbf{The corresponding code in the project is given in:\\}
		\vspace{2mm}
		\textit{Basic\_Calculations} \textnormal{in }\textit{BasicCalc.h}
		\vspace{2mm}
		
		\begin{verbatim}
		void Basic_Calculations(double PotentialDifference, double PlateDistance)
		{
		Energy_Electron = PotentialDifference / PlateDistance;  //Energy of the electron
		Force_Electron = -1 * ELECTRON_ENERGY * Energy_Electron; //Force on electron at Plate B
		Acceleration_Electron = fabs(Force_Electron) / ELECTRON_MASS;  //Acceleration of Electron
		}
		\end{verbatim}
		
		
		\subsection{Parallel to Electric Field}
			The velocity  \textbf{V} is given by :
			
			\begin{equation}
			V = V_0 + at
			\end{equation}
			where $V_0$ is the \textbf{initial velocity} of the electron or charged particle
			
			The displacement of the electron is given by 
			\begin{equation}
			S = V_0 + \frac{1}{2}at^2
			\end{equation}
			
			Velocity at any given time denoted by \textbf{$V_t$} is given by
			\begin{equation}
			V_t = \frac{eE}{2m}t = \sqrt{\frac{2eE}{m} x}
			\end{equation}
			
			\subsubsection{parallel electric field used in project}
			
			\begin{equation}
			X = \frac{eE}{2m}t^2
			\end{equation}
			
			The above equation gives the horizontal displacement of X which is depended
			on the value of \textbf{t}
			
			The calculation is done by the following code :
			\vspace{10mm}
			\begin{verbatim}
		void ElectronMovement_Parallel(int Identifier)
			{
			float count = Misc.count = 0;
			int Index = Misc.index =0;
			float Time = EField.Var.TimeEpoch;
			float StepSize = EField.Var.StepSize;
			unsigned int mem = Misc.MemAllocFactor;
			
			
			EField.CompArray.Xcomponent = (double *)calloc(mem, sizeof(double));
		
			while (count <= Time)
			{
			EField.Parallel.X_Component = fabs(Force_Electron) / (2 * ELECTRON_MASS) * pow(count, 2);
			
			EField.CompArray.Xcomponent[Index] = EField.Parallel.X_Component;
			count += StepSize;
			Index++;
			}
		
			}
			\end{verbatim}
	
	\subsection{Perpendicular to Electric Field}
	If \textbf{A} and \textbf{B} are two metal plates placed horizontally parallel to each other
	at a length \textbf{l}
	
	The velocity at \textbf{Y} axis would be 0 initially, and would be $V_0$ for \textbf{X} axis
	
	Therefore, \textit{$V_y$ = 0} and \textit{$V_x$ = $V_0$}.
	
	The  acceleration of the electron is given in y direction by 
	
	\begin{equation}
	\alpha_y = \frac{eE}{m}
	\end{equation}
	`
	As a result, the equation for velocity along the \textbf{Y} axis can be written as
	\begin{equation}
	V_y = \frac{eE}{m}t
	\end{equation}
	
	The horizontal, displacement \textbf{X} is given by the following equation
	\begin{equation}
		x = v_0t
	\end{equation}
	
	The time \textbf{t} is known as \textit{\textbf{transit time}} given by,
	\begin{equation}
	t = \frac{x}{v_0}
	\end{equation}
	
	If \textbf{$V_A$} is the accelerating potential, then the initial velocity $V_0$ is given by
	\begin{equation}
	V_0 = \sqrt{\frac{2eV_a}{m}}
	\end{equation}
	
	The vertical displacement can be written as 
	\begin{equation}
	y = Dtan\theta = D.\left({\frac{V_y}{V_0}}\right)
	\end{equation}
	
	The angular displacement is given by : 
	\begin{equation}
	\theta = tan^-1\left(\frac{V_y}{V_0}\right)
	\end{equation}
	
	
	
	\subsection{Electron Projected at an angle}
	Consider an electron projected at an angle $\theta_0$ to an electric field \textbf{E} 
	directed vertically downards, maintained between plates 
	\textbf{A}(positively charged plate and \textbf{B}) (negativelycharged plate)
		
		At the entry point $P_0$ the velocity $V_0$ has components 
		
		\begin{equation}
		V_{X0} = V_0 sin\theta_0 %need to check the equation if they are correct
		\end{equation}
		
		\begin{equation}
		V_{Y0} = V_0 cos\theta_0
		\end{equation}
	
	
	When the electron is pushed or is projected into the this field, the electron experiences 
	a force \textbf{f} which is directed to the positively charged plate. This causes the electron
	to have a projectile motion.\\
	
	The force \textbf{f} causes the electron to change its velocity from $v_1$ to $v_2$ at point $P_2$.
	
	At any instant the position of the electron canbe given by the following equations 
	
	\begin{equation}
	x = V_{X0} . t = \left(V_0cos\theta_0\right)t
	\end{equation}
	
	\begin{equation}
	y  =  V_{y0}  + \frac{1}{2}at^2 = (V_0 sin\theta_0)t + \frac{1}{2}at^2
	\end{equation}
	
	
	The maximum vertical displacement denoted by \textbf{H} of the electron can be calculated by 
	
	\begin{equation}
	H = \frac{{V_0}^2sin^2\theta_0}{2a}
	\end{equation}
	
	The time taken by the electron to achieve maximum displacement \textbf{t}
	is given by
	\begin{equation}
	t = \frac{v_0sin\theta_0}{a}
	\end{equation}
	
	The time of flight \textbf{T} gives the time taken by the electron
	to reach its final height
	
	\begin{equation}
	T = 2t = \frac{2V_0sin\theta_0}{a}
	\end{equation}
	
	
	The range \textbf{R} is gives the maximum horizontal displacement 
	
	\begin{equation}
	R = V_0 cos \theta_0\frac{2V_0(sin\theta_0)}{a}\\
	= \frac{\left(2sin\theta_0cos\theta_0\right){V_0}^2}{a}
	\end{equation}
	
	\begin{eqnarray}
	R = \frac{{V_0}^2 sin2\theta_0}{a}
	\end{eqnarray}
	
	\section{Motion Of an Electron inside a Magnetic Field}
	
	When a charge \textbf{q} enters a magnetic field at an angle \textbf{$\theta$} to the field intensity,
	it is acted upon by a magnetic force.
	
	\begin{equation}
	\vec{F} = \vec{qv} \times \vec{B}
	\end{equation}
	
	If an electron moves in an uniform magnetci field makin an angle $\theta$ to the field intensity, it is acted upon by magnetci force given by : 
	
	\begin{equation}
	\vec{F} = \vec{ev} \times \vec{B}
	\end{equation}
	If the velocity \textbf{v} and the intensity \textbf{B} are perpendicular to each other the magnetic
	force is given by 
	
	\begin{equation}
	F = ev B
	\end{equation}
	
	
  	\end{document}


